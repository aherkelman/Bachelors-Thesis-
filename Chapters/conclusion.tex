\chapter{Conclusion}
Using robotic manipulators to pick up objects requires methods of path planning and workspace visualization. The more degrees of freedom a manipulator has the simpler the design of a path planning algorithm is but the more robust the result becomes. As seen by the results of this project this manipulator with for DOF has a fairly limited workspace, which becomes much more so if orientation is desired and not just an xyz position on space. While 4 DOF is plenty to move this manipulator to a location it is not enough to be able to grab objects well since more orientation is needed. 

If grabbing objects is desired for a manipulator with a small degrees of freedom, such as in this project, it becomes necessary to limit the locations where objects can be placed to be picked up. Rather than placing the object to any place in the workspace it must be put somewhere the correct orientation can be achieved. While this limits more robust applications this sort of design could be used in a manufacturing situation where objects can be placed at more exact locations that would work for the arm's configuration. For use in a more broad setting with less controlled object locations a manipulator with a higher degree of freedoms would become necessary.

Even with a more maneuverable configuration the lack of feedback in the system would still cause a reliability issue. With so much error coming from the servos and camera the path planner can find a path that works appropriately but does not work successfully. To develop a reliable path planner feedback is crucial.  

This project showed 3D cameras as an accurate way to visualize a workspace. Without some form of stereo camera or a lengthy calibration process it is difficult to obtain 3D data about a color image. The depth sensing camera is a powerful tool to overcome the difficulty of collecting 3D data. Since depth data can be directly mapped to a color image it is possible to get 3D world coordinates for all points within a color image. This method is also fairly accurate to within about half an inch in every direction making this method of gathering workspace data very practical. 

Computer vision and path planning can be used together to survey and move a manipulator through an environment. While the path planning and data gathering sections can be accurate the biggest challenge becomes physically grabbing the object. Getting to the location is nor the biggest issue, that becomes placing the manipulator in such a position accurately enough to have a gripper close around it without any part of the gripper hitting the object too far away. With more precise servos and more degrees of freedom this task could certainly be accomplished.