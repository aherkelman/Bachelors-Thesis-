\chapter{Introduction}
 Robotics is becoming a more popular and useful field in the world today. Sensors have become more accurate and practical for mapping environments allowing robots to be used in more situations. Robotic manipulator's are used for situations where repetitive or dangerous tasks can be replaced with robotic . This is most commonly found in areas such as manufacturing. Combining manipulators with machine vision allows manipulators to adapt to different environments by visualizing the workspace.
 
 For this project a four degree of freedom robotic manipulator was used to grab objects within its workspace while avoiding other found obstacles. A three dimensional camera was used to gather data about the manipulator's workspace. Computer vision algorithms were developed to locate known objects within the camera's field of view. A path planning algorithm was then set up to calculate a path from the current arm position to a goal object and pick it up. 
 
 The algorithm used for path planning was a graph search method with an addition of a potential fields component to help the arm avoid obstacles while still moving towards the goal. The vision algorithms located goal objects and obstacles by using color and basic shape detection.  
 
 The report follows the following setup: Chapter 2 discusses different methods of path planning and object detection in images, Chapter 3 shows the results of the kinematic models of the manipulator, Chapter 4 goes over the hardware setup for this project, Chapter 5 states the results of developing image processing algorithms, Chapter 6 overviews another application for computer vision in robotics, Chapter 7 goes over the development of the path planning algorithm, Chapters 8 and 9 discuss the results of the path planing in simulation and real world implementation, Chapter 10 is the conclusion of the project.